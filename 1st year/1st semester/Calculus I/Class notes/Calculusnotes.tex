\documentclass[12pt, a4paper]{book}
\usepackage[top=2 cm, bottom=2 cm, left=2 cm, right=2 cm]{geometry}
\usepackage[T1]{fontenc}
\usepackage[utf8]{inputenc}
\usepackage{amsmath}
\usepackage{amssymb}
\usepackage{mathtools}
\usepackage{amsthm}
\usepackage[english]{babel}
\usepackage{parskip}
\usepackage{float}
\usepackage{physics}
\usepackage{graphicx}
\usepackage{multicol}
\usepackage{import}
\usepackage{cancel} 
\usepackage{amsthm}
\usepackage{xifthen}
\usepackage{pdfpages}
\usepackage{calc}
\usepackage{svg}
\usepackage{titling}
\usepackage{array}
\usepackage{enumitem}
\usepackage{yhmath}

\newcommand{\incfig}[1]{%
\def\svgscale{1}
\import{./Graphics/}{#1.pdf_tex}
}
\newcommand\vertarrowbox[3][6ex]{%
  \begin{array}[t]{@{}c@{}} #2 \\
  \left\downarrow\vcenter{\hrule height 0.2pt}\right.\kern-\nulldelimiterspace\\
  \makebox[0pt]{\scriptsize#3}
  \end{array}%
  }

\begin{document}
\theoremstyle{plain}
\newtheorem{thm}{Theorem}[chapter]
\newtheorem{lem}[thm]{Lemma}
\newtheorem{prop}[thm]{Proposition}
\newtheorem*{cor}{Corollary}


\theoremstyle{definition}
\newtheorem{defn}[thm]{Definition}
\newtheorem{conj}{Conjecture}[section]
\newtheorem{exmp}[thm]{Example}

\theoremstyle{remark}
\newtheorem*{rem}{Remark}
\newtheorem*{note}{Note}
\newtheorem*{notation}{Notation}
\newtheorem*{proposition}{Proposition}


\title{Calculus I}
\author{SyG}
\maketitle

\tableofcontents


\chapter{Introduction}
\begin{defn} \textbf{Mathematics (according to Oxford Eng. Dictionary)} \end{defn}
The abstract science which investigates deductively the conclusions implicit in the 
elementary conception of spacial and numerical relations.
This science can be divided in 6 main topics:
\begin{enumerate}
  \item \textbf{Foundations:} logic, set theory, proof theorems, etc.
  \item \textbf{Algebra:} numbers, arithmetical operations, order theorems. 
  \item \textbf{Analysis:} differentiation, integration, measure, etc.
  \item \textbf{Geometry and topology:} proporties of space, shape, position of figures.
  \item \textbf{Combinatories:} graph theory, partition theory, etc.
  \item \textbf{Applied Mathematics:} computational sciences, probability, the range of applications
  of Mathematics is wide, such as:
  \begin{enumerate}
    \item \textbf{Banking and Finance:} Black-Scholes equation.
    \item \textbf{Aeronautical engineering:} Fluid mechanics, shape design.
    \item \textbf{Chemistry:} Models for protein folding, thermodynamics.
    \item \textbf{Informatic:} Cryptography, computational algebra, parallel programming, etc. 
  \end{enumerate}
\end{enumerate}


\subsection*{Summary of the program: 1st Semester}
\begin{itemize}
  \item Real numbers
  \item Sequence and series of numbers
  \item Continuity and limit
  \item Differentiation
  \item Integration
\end{itemize}

\chapter{Real numbers and some basic concepts}

\section{Set of points}
We recall here some basic concepts:
\begin{defn}
  A \textbf{set} is a \textbf{collection of distinct objects.}
\end{defn}
\begin{exmp}
  2, 5, 7 are different objects (numbers). They can compose the set $\{2, \ 5, \ 7 \}$, where $\{ \ldots \}$ denotes the
  set composed by the objects $\ldots$.
\end{exmp}

\begin{note}
  If an object $x$ is a member of a set $\theta$, we denote: 
  $$ x \in \theta, \ \text{else we denote } x \notin \theta$$

  \begin{exmp}
    \[
      \theta = \{0, \ 5, \ 7 \}, \ \ \text{if }x=5 \text{ and } y=9:
    \]
    \[
      x \in \theta \text{ and } y \notin \theta
    \]
  \end{exmp}
\end{note}

\begin{rem}
  A set cannot have two times the same object.
\end{rem}

\begin{defn}
  Considering two sets A and B. If every element of A is a member of B, A is said to be a \textbf{subset} of B, and we denote:
  \boldmath $$ A \subseteq B$$, else we denote 
  $$ A \nsubseteq B $$ 
  Furthermore, if it exists at least one element of B which is not a member of A (A is strictly in B), A is said to be a \textbf{proper subset} of B, and we denote

  $$ A \subset B $$. \unboldmath
  
\end{defn}

\begin{exmp}
  \[
    A= \{ 1, \ 2, \ 3\} 
  \]
  \[
    B= \{ 0, \ 1, \ 2, \ 3, \ 4 \}
  \]
  \[
    C= \{ 0, \ 1, \ 2 \}
  \]
  \[
    \therefore
    A \subseteq A, \ \ \ A \subset B, \ \ \ A \nsubseteq C
  \]
\end{exmp}

\begin{defn}
  \textbf{Set operators}
  Let A and B be two sets. 
\end{defn}
\boldmath
\subsubsection*{Union: $\cup$}

The \textbf{union} of $A$ and $B$ is the set

\[ 
  A \cup B = \{x\vert x \in A \vee x \in B \}
\]

\subsection*{Intersection: $\cap$}

The \textbf{intersection} of $A$ and $B$ is the set
\[
  A \cap B = \{x\vert x \in A \wedge x \in B \}
\]

\subsection*{Complement: $\backslash$}

\[
  A \; \backslash \; B = \{x\vert x \in A \wedge x \notin B \}
\]
\unboldmath
\begin{exmp}
  $A=\{3,5,7\}, B=\{5,7,10\}$
  \begin{itemize}
    \item $A \cup B = \{3,5,7,10\}$
    \item $A \cap B = \{5,7\}$
    \item $A \backslash B = \{3\}$
    \item $B \backslash A = \{10\}$
  \end{itemize}
\end{exmp}

\begin{rem}
  A set of one element is called a \textbf{singleton}.
\end{rem}

\subsubsection*{Geometrical representation}

\begin{figure}[H]
  \centering
  \incfig{Graphic1c}
\end{figure}

\subsubsection*{Properties (Morgan for sets): }

\begin{itemize}
  \item $A \backslash (B \cup C) = (A \backslash B) \cup (A \backslash C)$
  \item $A \backslash (B \cap C) = (A \backslash B) \cap (A \backslash C)$
\end{itemize}

Graphically (Venn diagrams):

\begin{figure}[H]
  \centering
  \incfig{Graphic2a}
\end{figure}

\begin{figure}[H]
  \centering
  \incfig{Graphic2b}
\end{figure}

\begin{note}
  In the case of various $n$ sets denoted by $A_1, \ A_2, \ A_n$, instead of writing:

  \[
    A_1 \cup A_2 \cup \ldots \cup A_n \text{ we write } \bigcup_{k=1}^{n} A_k
  \]
  or
  \[
    A_1 \cap A_2 \cap \ldots \cap A_n \text{ we write } \bigcap_{k=1}^{n} A_k
  \]
\end{note}

\begin{exmp}
  \[
    A_1=\{1, \ 2, \ 3 \}, \ \ \ A_2=\{ 5,\ 6, \ 7 \}, \ \ \ A_3= \{1, \ 5, \ 9 \}
  \]
  \[
    \bigcup_{k=1}^{3} A_k=A_1 \cup A_2 \cup A_3 = \{ 1, \ 2, \ 3, \ 5, \ 6, \ 7, \ 9 \}
  \]
\end{exmp}

\begin{rem}
We can apply the same notation in case of infinite ($\infty$) numbers of a set $\{ A_1, \ldots, A_{100}, \ldots \}$.
\[
  \displaystyle\bigcup_{k=1}^{\infty}A_k \ \ \ \ \text{ and } \ \ \ \ \displaystyle\bigcap_{k=1}^{\infty}A_k
\]
\end{rem}
some examples and the concept of infinity will be defined in the next sections.

\begin{defn}
  The cartesian product of two sets A and B is denoted by \boldmath $A \times B$  and defined as:
  \[
    A \times B = \{ (a, \ b  ) \ | \ a \in A \text{ and } b \in B \}
  \]
  \unboldmath
  where $(a, b)$ is called \textbf{ordered pair.} 
\end{defn}

\begin{exmp}
  \[
    A=\{1, \ 2, \ 3 \}, \ \ \ B=\{7, \ 9 \}
  \]
  \[
    A \times B= \{ (1,7), \ (1,9), \ (2,7), \ (2,9), \ (3,7), \ (3,9) \}
  \]
  The order is very important, it always goes first the elements of the first named set and then the ones of the second one.More properties of sets will be introduced later in this chapter.
\end{exmp}

\subsection*{Some common sets of real points}
Here we only introduce the set of points used in next chapters.

\begin{defn}
\end{defn}
\boldmath
  \begin{itemize}
    \item $\mathbb{R}=\{ \ldots, \ \ldots, \ -10, \ \ldots, \ -7, \ \ldots, \ 0, \ \ldots, \ 4, \ \ldots, 1000, \ \ldots  \}$ is called the set of \textbf{real numbers} which contains \textbf{all positive and negative numbers}.
    \item $\mathbb{N}= \{1, \ 2 , \ 3 , \ldots \}$ is called the set of \textbf{natural numbers} which contains \textbf{all the strictly positive integer numbers.}
    \begin{rem}
      $\mathbb{N^{*}}$ includes the 0.
    \end{rem}
    \item $\mathbb{Z}= \{\ldots, -5, \ -4, \ -3, \ -2, \ -1, \ 0, \ 1, \ 2, \ 3, \ldots  \}$ is called the set of \textbf{integer numbers} and contains the \textbf{positive and negative integers.}
    \item $\varnothing = \{\}$ the \textbf{empty set} represents the sets without any elements. 
    \begin{exmp}
      \[
        A=\{1, \ 4 \}, \ \ B=\{3, \ 4 \} \ \ \ \ A \cap B= \varnothing \text{, ie no coincidences between A and B}
      \]
    \end{exmp} 
    \item $\mathbb{Q}= \{x \in \mathbb{R} \ \vert \ x= \frac{m}{n}, \ m \in \mathbb{Z}, \ n \in \mathbb{Z} \text{ and } n \neq 0 \}$ is called the set of \textbf{rational numbers} and contains the \textbf{real numbers that can be written as a quotient of integer numbers}
    Numbers that don't belong in this set, ie $\sqrt{2}$ or $\pi$ are part of the \textbf{irrationals}, preferably noted as $\notin \mathbb{Q}$.
    \item \textbf{Odd=} $\{ x \in \mathbb{R} \ \vert \ \exists \ k \in \mathbb{N} \text{ st } x=2k+1 \}$ is the set of the \textbf{odd integer numbers}.
    \item \textbf{Even=} $\{ x \in \mathbb{R} \ \vert \exists \ k \ \in \mathbb{N} \text{ st } x=2k \}$ is the set of the \textbf{even integer numbers}.
    \item $\mathbb{C}= \{x +iy \ \vert \ x \in \mathbb{R} \text{ and } y \in \mathbb{R}  \}$ is the set of \textbf{complex numbers}. Note: i denotes the imaginary number that verifies $i^2=-1$.
  \end{itemize}
  \unboldmath

\begin{rem}
  \[
    \varnothing \subseteq \mathbb{N} \subseteq \mathbb{Z} \subseteq \mathbb{Q} \subseteq \mathbb{R} \subseteq \mathbb{C}
  \]
\end{rem}

\begin{defn}
  Let $A$ and $B$ be two sets, and their cartesian product $A \times B$, any subset $S \subseteq A \times B$ is called a \textbf{relation} between $A$ and $B$.
\end{defn}

\begin{exmp}
  $A = \{1,2,3\}, B = \{a,b\}$. A relation could be $R = \{(1,a), (2,b), (3,b)\}$.
\end{exmp}

\begin{defn}
  Some properties of relation $S$ are:
  \begin{itemize}
    \item \textbf{Reflexive:} $\forall a \in S, \ (a,a)\in S$
    \item \textbf{Symetric:} $(a,b)\in S$ and $ (b,a)\in S \Rightarrow \ a=b$
    \item \textbf{Antisymmetric:} $(a,b)\in S$ and $ (b,a)\in S \Rightarrow \ a=b$
    \item \textbf{Transitive:} $(a,b)\in S$ and $(b,c)\in S \Rightarrow (a,c)\in S$
    \item \textbf{Comparable (or connex):} $\forall \ a, b \in S$, either $(a,b)\in S$ or $(b,a)\in S$
  \end{itemize}
  \textbf{Important note}: antisymmetric \textbf{IS NOT} the negation of symetric. The negation of symetric would be asymetric.
\end{defn}

\begin{defn}
  \boldmath
  A \textbf{relation}, noted by $\leq $, is a \textbf{total order} on a set S if it verifies:
  \begin{enumerate}
    \item \textbf{Reflexivity:} $\forall a \in S, \ a \leq a$
    \item \textbf{Antisymmetry:} $a \leq b$ and $ b \leq a \Rightarrow \ a=b$
    \item \textbf{Transitivity:} $a \leq b$ and $b \leq c \Rightarrow a \leq c$
    \item \textbf{Comparability:} $\forall \ a, b \in S$, either $a \leq b$ or $b \leq a$
  
    When Reflexivity, Antisymmetry and Transitivity occurs but no Comparability, then we have a \textbf{partial order}. For a \textbf{total order}, we would need Comparability.
  \end{enumerate}
  \unboldmath
\end{defn}

\begin{exmp}
  The divisibility relation, denoted by $``|''$, on the set of natural numbers $\mathbb{N} = \{1,2,3,\ldots\}$ is a classic example of a partial order relation.
  \begin{itemize}
    \item The relation $``|''$ is reflexive, because any $a\in\mathbb{N}$ divides itself.
    \item The relation $``|''$ is antisymmetric. Indeed, if $a|b$, then $ak=b$, where $k$ is an integer. Similarly, if $b|a$, then $bl=a$, where $l$ is an integer. Hence, $akl=l\Rightarrow kl=1$. This last equation only holds if $k=l=1$, which means that $a=b$.
    \item The relation $``|''$ is transitive. Suppose $a|b$ and $b|c$. Then $ak=b$ and $bl=c$, where $k$, $l$ are certain integers. Hence $akl=c$, and $kl$ is an integer. That means $a|c$.
  \end{itemize}
\end{exmp}

\begin{exmp}
  \begin{itemize}
    \item The relation $\leq$ applied to $\mathbb{R}$ is a total order.
    \item The relation $\subset$ applied to a subset of $\mathbb{R}$ is \textbf{not} a total order. For example, $\{1, \ 2\}$ and $\{2, \ 4\}$ cannot be compared.  
  \end{itemize}
\end{exmp}

\begin{defn}
  A set plus a total order relation is called a \textbf{total ordered set}.
\end{defn}

\begin{exmp}
  \[
    (\mathbb{R}, \leq)
  \]

  Rules:
  \begin{enumerate}
    \item $a=b$
    \item $a<b$ or $a>b$ a strictly inferior (or superior) to b (not equal).
  \end{enumerate}
\end{exmp}

\begin{defn}
  \textbf{Infinity:} denoted by $\infty$, is an abstract concept ??? a limitless quantity (e.g. number).
\end{defn}

\subsubsection*{Properties:}

\begin{itemize}
  \item $\forall x \in \mathbb{R}, \ \ - \infty \leq x$ and $x \leq + \infty \ \therefore \ \mathbb{R}=(-\infty, +\infty).$
  \item $-\infty$ and $+\infty \notin \mathbb{R}$
\end{itemize}

\begin{defn}
  An \textbf{interval} is a real subset containing all the values between two given points, included or not. It can be of the type: 
  
  Let $a, \ b \in \mathbb{R}$:
  \begin{itemize}
    \item \textbf{Open interval}: $(a,b) = \{x\in \mathbb{R} \vert a < x < b\}$
    \item \textbf{Closed interval}: $[a,b] = \{x\in \mathbb{R} \vert a \leq x \leq b\}$
    \item \textbf{Left closed interval}: $[a,b) = \{x\in \mathbb{R} \vert a \leq x<b\}$
    \item \textbf{Left open interval}: $(a,b] = \{x\in \mathbb{R} \vert a < x \leq b\}$
  \end{itemize}
\end{defn}

\subsubsection*{Properties:} $\mathbb{R} = (-\infty , +\infty)$

\subsection*{Graphical representation:}

\begin{figure}[H]
  \centering
  \incfig{Graphic3}
\end{figure}

\begin{figure}[H]
  \centering
  \incfig{Graphic3b}
\end{figure}

\begin{figure}[H]
  \centering
  \incfig{Graphic3c}
\end{figure}

\begin{figure}[H]
  \centering
  \incfig{Graphic3d}
\end{figure}

\begin{defn}
  \textbf{Axiomatic definition of }$\mathbb{R}$.

  The real number system $(\mathbb{R}, +, \cdot, <)$ is a set where the following rules are defined.

  \begin{itemize}
    \item \textbf{Addition (+)}: a function
    \begin{align*}
      \mathbb{R}\times \mathbb{R} &\mapsto \mathbb{R} \\
      (x,y) &\mapsto x+y
    \end{align*}
    with the following properties:
    \begin{itemize}
      \item \textbf{Associativity}: $\forall x,y,z \in \mathbb{R}, \ (x+y)+z=x+(y+z)$
      \item \textbf{Commutativity}: $\forall x,y \in \mathbb{R}, \ x+y=y+x$
      \item \textbf{Identity element}: $\exists 0 \in \mathbb{R}\vert\ 0+x=x+0=x$
      \item \textbf{Opposite element}: $\forall x \in \mathbb{R}, \exists! -x\in\mathbb{R}\vert \ x+(-x)=(-x)+x=0$
    \end{itemize}

    \item \textbf{Multiplication ($\cdot$)}: a function 
    \begin{align*}
      \mathbb{R}\times \mathbb{R} &\longrightarrow \mathbb{R} \\
      (x,y) &\longrightarrow x \cdot y
    \end{align*}
    with the following properties:
    \begin{itemize}
      \item \textbf{Associativity}: $\forall x,y,z \in \mathbb{R}, \ (xy)z=x(yz)$
      \item \textbf{Commutativity}: $\forall x,y \in \mathbb{R}, \ xy=yx$
      \item \textbf{Identity element}: $\exists 1 \in \mathbb{R}\vert \ 1\cdot x=x\cdot 1=x$
      \item \textbf{Inverse element}: $\forall x \in \mathbb{R}, \exists! \frac{1}{x}\in\mathbb{R}\vert \ x\cdot \frac{1}{x} = \frac{1}{x} \cdot x = 1$
      \item \textbf{Distributivity}: $\forall x,y,z \in \mathbb{R}\backslash \{0\}, \ x(y+z)=xy + xz$
    \end{itemize}

    \item The field $(\mathbb{R},+,\cdot)$ is ordered:
    \begin{itemize}
      \item $\geq$ is a total order.
      \item $\forall x,y,z\in\mathbb{R}, \ x\geq y \Rightarrow x+y \geq y+z$
      \item $\forall x,y \geq 0, \ xy\geq 0$
    \end{itemize}

    \item The order is \textbf{Dedekind complete} (the supremum property):
    
    $A \neq \emptyset, \ A \subseteq \mathbb{R} \wedge \exists k\in\mathbb{R} \vert \forall a\in A, a\leq k$ (where $k$ is called \textit{upper bound}) $\Rightarrow \exists \alpha$ denoted $\sup A$ and called least upper bound, such that $\forall a\in\mathbb{A}, a\leq\alpha$ and $\forall k\in\mathbb{R}$ upper bound of $A,\ \alpha \leq k$.
  \end{itemize}
\end{defn}

\begin{rem}
  $\mathbb{N}$ cannot be defined axiomatically (e.g. $0\notin\mathbb{N}$)
\end{rem}


\section{Mathematical Functions}

\begin{defn}
  \boldmath Let $A$ and $B$ being two sets. A function from $A$ to $B$, is a relation between $A$ and $B$, denoted by $f:A\rightarrow B$, such that $\forall a\in A, \ \exists!\ b\in B / \ f(a) = b$. \unboldmath

  The elements of $A$ are called \textbf{arguments of} $f$. The element $b\in B$ such that $f(a) = b$, with $a \in A$ is called \textbf{value} at $a$ or \textbf{image} of $a$ under $f$.

  $A$ is called \textbf{domain} of $f$, $D(f)$, and $R(f) = \{b\in B\vert\ \exists a\in A \vert\ f(a)=b\}$ is the \textbf{range}.
\end{defn}

\begin{notation}
  \begin{align*}
    f:A &\longrightarrow B \\
    a &\longrightarrow f(a)
  \end{align*}
  We can write \boldmath $f$ \textbf{maps} $A$ to $B$. \unboldmath
\end{notation}

\begin{exmp}
  \begin{align*}
    f:\mathbb{R} &\longrightarrow \mathbb{R} \\
    x &\longrightarrow x^2+1
  \end{align*}
  $f(1) = 1^2+1 = 2$

  $f(2) = 2^2+1 = 5$

  \begin{figure}[H]
    \centering
    \incfig{Graphic4a}
  \end{figure}

  \begin{figure}[H]
    \centering
    \incfig{Graphic4b}
  \end{figure}
  
\end{exmp}

\begin{defn}
  The graph of a function is \textbf{its set of ordered pairs} \boldmath $F=\{(a,f(a)),\ \forall a\in A\}$
  \begin{align*}
    f:\mathbb{R} &\longrightarrow \mathbb{R} \\
    x &\longrightarrow x^2+1
  \end{align*}
  \unboldmath

\end{defn}

\begin{figure}[H]
  \centering
  \incfig{Graphic5}
\end{figure}

\begin{defn}
  \begin{itemize}
    \item If $E \subseteq A$, the image of $E$ under $f$ is $f(E) = \{f(x)\vert\ x\in E\}$.
    \item If $H \subseteq B$, the preimage of $H$ under $f$ is $f^{-1}(H) = \{x\in A\vert\ f(x)\in H\}$
  \end{itemize}
\end{defn}

\begin{exmp}
  $A=B=\mathbb{R},\ f(x) = x^2$.
  \begin{itemize}
    \item $E = [0,2] \subset \mathbb{R} \quad f(E) = [0,4]$
    \item $H = \{4,9\} \subset \mathbb{R} \quad f^{-1}(H) = \{-3,-2,2,3\}$
  \end{itemize}
\end{exmp}

\begin{defn}
  \boldmath
  $f:A\longrightarrow B$
  \begin{itemize}
    \item $f$ is called \textbf{inyective} if $\forall a_1,a_2 \in A, f(a_1)=f(a_2) \Rightarrow a_1=a_2$
    \item $f$ is called \textbf{suryective} if $\forall b \in B,\ \exists a\in A \vert\ f(a)=b$
    \item $f$ is called \textbf{biyective} if $f$ is \textbf{inyective and suryective.}
  \end{itemize}
  \unboldmath
\end{defn}

\begin{rem}
  If a function is biyective you can obtain its inverse.
\end{rem}

\begin{exmp}
  \begin{itemize}
    \item 
    \begin{align*}
      f:\mathbb{R} &\longrightarrow \mathbb{R} \\
      x &\longrightarrow x^2
    \end{align*}
    It's not inyective, since $f(1) = f(-1)$, and it's not suryective since $-1\in\mathbb{R} \wedge \nexists a \in \mathbb{R}\vert\ f(a)=-1$

    \item 
    \begin{align*}
      f:[0,1] &\longrightarrow [1,2] \\
      x &\longrightarrow x+1
    \end{align*}
    It is inyective, since $f(x_1) = f(x_2) \Rightarrow x_1+1=x_2+1 \Rightarrow x_1 = x_2$. It is also suryective, since $\forall x_1 \in [1,2],\; x_2=x_1-1 \in [0,1]$ and $f(x_2) = x_1$. Thus $f$ is biyective.
  \end{itemize}
\end{exmp}

\begin{defn}
  \boldmath
  Let $f:A \longrightarrow B$ and $f:B \longrightarrow C$, the \textbf{composition of} $f$ with $g$ is a function denoted as $g \circ f$ and defined by:
  \begin{align*}
    g \circ f:A &\longrightarrow C \\
    a &\longrightarrow g(f(a))
  \end{align*}
  \unboldmath
\end{defn}

\begin{exmp}
  \begin{equation*}
    \begin{aligned}[c]
      g:\mathbb{R} &\longrightarrow \mathbb{R} \\
      x &\longrightarrow \cos (x)
    \end{aligned}
    \qquad \qquad
    \begin{aligned}[c]
      f:\mathbb{R} &\longrightarrow \mathbb{R} \\
      x &\longrightarrow x^2
    \end{aligned}
  \end{equation*}
  $g o f(x) = \cos (x^2) \neq f o g(x) = {(\cos(x))}^2$
\end{exmp}

\begin{defn}
  \boldmath
  The \textbf{identity function} $f$ on $A$, is the function:
  \begin{align*}
    {id}_A:A &\longrightarrow A \\
    x &\longrightarrow x
  \end{align*}
  \unboldmath
\end{defn}

\begin{defn}
  \boldmath
  Let $f:A\longrightarrow B$ (biyective). The inverse function of $f$, denoted by $f^{-1}$, is the function $f^{-1}:B \longrightarrow A$ such that:
  \[
    b = f(a) \Leftrightarrow a = f^{-1}(b)
  \]
  Note that this is the same as saying $f \circ f^{-1}(b) = {id}_B$ and $f \circ f^{-1}(a) = {id}_A$.
  \unboldmath
\end{defn}

\begin{exmp}
  $f o f^{-1}(x) = f(x-1) = (x-1)+1 = x$ and $f^{-1} o f(x) = (x+1)-1 = x$.
\end{exmp}

\begin{defn}
  We call \textbf{real function} with real variable a function of the type $f:\mathbb{R} \longrightarrow \mathbb{R}$
\end{defn}

\section{Some properties of particular real sub-sets}

\subsection{Odd and Even sets}

\begin{proposition}
  \boldmath
  Let $O_1,O_2 \in$ Odd, and $e_1,e_2 \in$ Even.
  \begin{enumerate}[label=\emph{\alph*})]
    \item $O_1 + O_2$ is even.
    \item $e_1 + e_2$ is even.
    \item $e_1 + O_1$ is odd.
    \item $e_1 \cdot e_2$ is even.
    \item $O_1 \cdot O_2$ is odd.
    \item $e_1 \cdot O_1$ is even.
  \end{enumerate}
  \unboldmath
\end{proposition}

\begin{proof}
  Let $e_1 = 2 k_1, e_2 = 2 k_2, O_1 = 2 k_3+1, O_2 = 2 k_4+1$ with $k_1, k_2, k_3, k_4 \in\mathbb{Z}$.
  \begin{enumerate}[label=\emph{\alph*})]
    \item $O_1 + O_2 = 2 k_3 +1 +2 k_4 +1 = 2(k_3+k_4) + 2 = 2(k_3+k_4+1) \in$ Even.
    \item $e_1 + e_2 = 2 k_1 + 2 k_2 = 2(k_1+k_2) \in$ Even.
    \item $e_1 + O_1 = 2 k_1 +2 k_3 +1 = 2 (k_1+k_3) +1 \in$ Odd.
    \item $e_1 \cdot e_2 = 2 k_1 \cdot 2 k_2 = 2(2k_1 k_2) \in$ Even.
    \item $e_1 \cdot O_1 = 2 k_1 \cdot (2 k_3 +1) = 2 k_1 k_3 + 2 k_1 = 2 (k_1 k_3 + k_1) \in$ Even.
  \end{enumerate}
\end{proof}

\begin{proposition}
  \begin{enumerate}[label=\emph{\alph*})]
    \item $n$ is even from previous proposition. $n^2 = n \cdot n$ is even.
    \begin{proof}
      Trivial due to section e) of previous proofs.
    \end{proof}
    \item \begin{proof}
      ${(n+p)}^2$ is even $\Rightarrow (n+p)$ is even $\Rightarrow$ $n,p$ are even $\vee\ n,p$ are odd $\Rightarrow$  
    
      $\Rightarrow \displaystyle
      \begin{cases} 
        \text{if } n,p \text{ even} &\Rightarrow n-p \text{ is even} \\
        \text{if } n,p \text{ odd} &\Rightarrow n-p \text{ is even}
      \end{cases}\quad \Rightarrow {(n-p)}^2$ is even.

      $\Leftarrow$ would use the same idea. Justifying steps with previous proof.
    \end{proof}
    % Aca habian varias pruebas redundantes (?????)
  \end{enumerate}
\end{proposition}

\subsection{$\mathbb{N}$ and $\mathbb{Z}$}

\begin{prop}
  Let $n_1, n_2 \in \mathbb{N}$ and $z_1, z_2 \in \mathbb{Z}$
  \begin{enumerate}[label=\emph{\alph*})]
    \item $n_1 + n_2 \in \mathbb{N}$
    \item $n_1 \cdot n_2 \in \mathbb{N}$
    \item $z_1 + z_2 \in \mathbb{N}$
    \item $z_1 \cdot z_2 \in \mathbb{N}$
    \item $n_1 \geq n_2$ or $n_2 \geq n_1$
    \item $z_1 \geq z_2$ or $z_2 \geq z_1$
  \end{enumerate}
\end{prop}

\begin{prop}
  \textbf{Well-Ordering Principle}
  \boldmath
  Let $B \subseteq \mathbb{N}$ and $B \neq \emptyset$.

  It always exists $n_0 \in B$ such that $\forall m\in B$, $n_0 \leq m$. Such $n_0$ is called minimum of $B$ and denoted $\min B$.
  *** principio pagina 13
  \unboldmath
\end{prop}

\begin{defn}
  \textbf{Mathematical Induction}

  We want to demonstrate a statement $P_n$ involving $n\in\mathbb{N}$ for all values of $n$. We have to follow these steps:
  \begin{enumerate}[label=\emph{\alph*})]
    \item We prove that the statement holds for the first value of $n$.
    \item We prove that if the statement holds for $n$, then it holds for $n+1$.
  \end{enumerate}
\end{defn}

\begin{exmp}
  \begin{proof}
    \textbf{W.O.P.} (Well-Ordering Principle). We will see the following proposition is false, proving it by absurdity.

    Let $J = \mathbb{N} \backslash B$. $"P_n = \{1,\ldots,n\} \in J"$. We start with
    \begin{enumerate}[label=\emph{\alph*})]
      \item $P_1 = ``1 \in J''$. True, else we would be saying that $\min B = 1$.
      \item $P_n = \{1,\ldots,n \in J\}$, then $n+1 \in J$, else $\min B = n+1$ (as $1,\ldots,n \notin B$).
    \end{enumerate}
    $\Rightarrow \forall n\in\mathbb{N}\in J \Rightarrow B = \emptyset \Rightarrow$ ABSURD.
  \end{proof}
\end{exmp}

\begin{exmp}
  \begin{enumerate}[label=\emph{\alph*})]
    \item $\displaystyle P_n = \sum_{k = 1}^{n}k = \frac{n(n+1)}{2}$
    \begin{itemize}
      \item $\displaystyle P_1 = 1 = \frac{1(1+1)}{2} = 2$
      \item $P_n$ true. $\displaystyle P_n = \sum_{k = 1}^{n}k = \frac{n(n+1)}{2} \Rightarrow \sum_{k = 1}^{n+1}k = \frac{n(n+1)}{2} + n+1 = \frac{n(n+1)}{2} + n+1 = \frac{n(n+1)+2n+2}{2} = \frac{(n+1)((n+1)+1)}{2} \Rightarrow \sum_{k = 1}^{n+1}k = \frac{(n+1)((n+1)+1)}{2}$ $P_{n+1} =$ true.
    \end{itemize}
    \item $\displaystyle P_n = \sum_{k = 0}^{n}r^k = \frac{1-r^{n+1}}{1-r}$
    \begin{itemize}
      \item $\displaystyle P_1 = r^0 + r = 1+r = \frac{1-r^2}{1-r} = \frac{(1+r)(1-r)}{1-r} = 1+r$ true.
      \item $P_n$ true. $\displaystyle P_n = \sum_{k = 0}^{n+1}r^k = \sum_{k = 0}^{n}r^k + r^{n+1} = \frac{1-r^{n+1}+r^{n+1}-r\cdot r^{n+1}}{1-r} = \frac{1-r^{n+2}}{1-r} \Rightarrow P_{n+1}$ true.
    \end{itemize}
  \end{enumerate}
\end{exmp}


\section{$\mathbb{Q}$}

\begin{proposition}
  $Q_1,Q_2 \in \mathbb{Q}$
  \begin{enumerate}[label=\emph{\alph*})]
    \item $Q_1 + Q_2 \in \mathbb{Q}$
    \item $Q_1 \cdot Q_2 \in \mathbb{Q}$
    \item $\displaystyle \frac{1}{Q_1} \in \mathbb{Q}$ (if $Q_1 \neq 0$)
    \item $Q_1 \leq Q_2$ or $Q_2 \leq Q_1$
  \end{enumerate}
\end{proposition}

\begin{exmp}
  a) Let $\displaystyle p \in \mathbb{Q}, p\neq 0 \Rightarrow p = \frac{a}{b}, a,b \in \mathbb{Z}\backslash \{0\}$.

  Let $\displaystyle x \notin \mathbb{Q}$. If $p+x \in \mathbb{Q}$, $\exists c,d \in \mathbb{Z} \vert p+x = \frac{c}{d} \Rightarrow \frac{a}{b} + x = \frac{c}{d} \Rightarrow x = \frac{c}{d} - \frac{a}{b} = \frac{bc-ad\in \mathbb{Z}}{db \in \mathbb{Z}\backslash \{0\}} \Rightarrow x \in \mathbb{Q}$ absurd.
\end{exmp}

\begin{exmp}
  $\sqrt{2}$ is irrational.
  \begin{proof}
    Lets assume that $\sqrt{2}$ is rational. Then we can write $\displaystyle\sqrt{2} = \frac{p}{q}$ where we may assume that $p$ and $q$ have no common factors (if there are any common factors we cancel them in the numerator and denominator) $\Rightarrow \frac{p^2}{q^2} = 2 \Rightarrow p^2 = 2 q^2 \Rightarrow p$ even $\Rightarrow p^2$ is divisible by $4$ (i.e. $p^2=4m, m\in\mathbb{N}$) $\Rightarrow q^2 = 2m \Rightarrow q$ even (i.e. $q = 2s, s\in\mathbb{N}$). Then $p$ and $q$ have a common factor ($2$). Absurd.
  \end{proof}
\end{exmp}

\begin{prop}
  $\sqrt{m}$, with $n\in\mathbb{N}$ and such that $n$ is not a square number (i.e. $\nexists\ p\in\mathbb{N} \vert\ p^2 = n$), is irrational.
\end{prop}

\begin{defn}
  An \textbf{algebraic number} is a real number that is a root of a non-zero polinomial with rational coefficients, i.e. if $n$ is algebraic, it exists a polinomial
  \[
    \displaystyle p(x) = a_0 + a_1 x + a_2 x^2 + \ldots + a_k x^k = \sum_{i = 0}^{k}a_i x^i  
  \]
  such that $p(n)=0$.
\end{defn}

\begin{exmp}
  \begin{enumerate}[label=\emph{\alph*})]
    \item If $\sqrt{2} - \sqrt{3}$ is rational $\Rightarrow {(\sqrt{2} - \sqrt{3})}^2$ is rational $\Rightarrow 5 - 2\sqrt{2}\sqrt{3}$ is rational $\Rightarrow \sqrt{6}$ is rational $\Rightarrow$ absurd.
    \item $\displaystyle 1 - \sqrt[3]{2+\sqrt{5}}$
    
    \begin{proof}
      $2$ is algebraic and $\sqrt{5}$ is algebraic (since it exists the polinomial $x^2-5$) $\Rightarrow 2+\sqrt{5}$ is algebraic and root of $p(x) \Rightarrow \sqrt[3]{2+\sqrt{5}}$ is root of $p(x^3)$ (i.e. all exponents of $x$ are increased by $3$). Then $p(x)=x^2+x-1 \Rightarrow p(x^3) = x^6 +x^3 -1 \Rightarrow 1$ is algebraic $\Rightarrow \sqrt[3]{2+\sqrt{5}}$ is algebraic.
    \end{proof}
  \end{enumerate}
\end{exmp}

\begin{defn}
  A decimal representation of a real number $r$ is an expression of the form:
  \[
    r = a_0. a_1 a_2 a_3 a_4 \ldots \text{ with } a_i \in\mathbb{N}, i\in\mathbb{N}
  \]
  $\displaystyle r = \sum_{i = 1}^{\infty} \frac{a_i}{10^i} $
\end{defn}

\begin{prop}
  The decimal representation of a real number $r$ either terminates (i.e. $\exists n_0\in\mathbb{N}\ \vert\ a_i=0\ \forall i \geq n_0$) or begins to repeat a same finite sequence over and over iff $r$ is rational.
\end{prop}

\begin{exmp}
  $0.127841841841841\ldots$ is rational.

  $\pi = 3.14159265\ldots$ is irrational.
\end{exmp}

\begin{prop}
  $\mathbb{Q}$ and $\mathbb{R}\backslash\mathbb{Q}$ are dense in $\mathbb{R}$: $\forall x_1,x_2 \in\mathbb{R}, x_1<x_2,\ \exists q\in\mathbb{Q}$ and $n\in\mathbb{R}\backslash\mathbb{Q}\ \vert\ x_1<q<x_2$ and $x_1<n<x_2$.
\end{prop}

\section{$\mathbb{R}$}

\subsection*{Distance in $\mathbb{R}$}

\begin{defn}
  The \textbf{absolute value} or a modulus of a real number $x$ is denoted $|x|$ and defined by:
  \[
    |x| = \begin{cases}
      x &x\geq 0 \\
      -x &x<0
    \end{cases}
  \]
\end{defn}

\begin{figure}[H]
  \centering
  \incfig{Graphic7}
\end{figure}

\begin{prop}
  Properties
  \begin{enumerate}[label=\emph{\alph*})]
    \item $|x| = \sqrt{x^2}$
    \item $|xy| = |x|\cdot |y|$
    \item $|x+y| \leq |x| + |y|$
  \end{enumerate}
  \begin{proof}
    $\begin{cases}
      x\leq |x| & \wedge\ y\leq |y| \\
      -x\leq |x| & \wedge\ -y\leq |y|
    \end{cases} \Rightarrow 
    \begin{cases}
      x+y \leq |x|+|y| \\
      -(x+y) \leq |x|+|y|
    \end{cases} \Rightarrow
    |x+y|\leq |x|+|y|$
  \end{proof}
\end{prop}

\begin{defn}
  $\forall x,y\in\mathbb{R}$ we call $|x-y|$ the distance from $x$ to $y$.
\end{defn}

\begin{proposition}
  Properties. $\forall x,y,z \in\mathbb{R}$
  \begin{enumerate}[label=\emph{\alph*})]
    \item $|x-y|=0 \Leftrightarrow x=y$
    \item $|x-y| = |y-x|$
    \item $|x-y| \leq |x-z| + |z-y|$ (Triangle inequality)
  \end{enumerate}
\end{proposition}

\begin{exmp}
  \begin{enumerate}[label=\emph{\alph*})]
    \item $|2x+3|-1 < |x|$
    We determine all the cases of signs:
    \[
      \displaystyle x < -\frac{3}{2}, 0>x\geq -\frac{3}{2}, x\geq 0
    \]
    \begin{itemize}
      \item $x\geq 0 \Rightarrow |2x+3| = 2x+3 \geq 0$ and $|x|=0\geq 0$. The equation leads to:

      $2x+3-1 < x \Rightarrow x < -2 \Rightarrow x < -2 \Rightarrow$ absurd.
      \item $x<-\frac{3}{2} \Rightarrow |2x+3|=-2x-3$ and $|x|=-x$. Then $-2x-3-1 < -x \Rightarrow -2x-4 < -x \Rightarrow 2x+4>x \Rightarrow x>-4 \Rightarrow x\in(-4,-\frac{3}{2})$
      \item $x\in[-\frac{3}{2},0) \Rightarrow |2x+3|=2x+3,\ |x|=-x$

      $2x+3-1 < -x \Rightarrow 3x < -2 \Rightarrow x< -\frac{2}{3} \Rightarrow x\in[-\frac{3}{2}, -\frac{2}{3}]$
    \end{itemize}
    The solutions are $\displaystyle x\in(-4,-\frac{2}{3})$

    \item Same idea: $x\geq 4;\ 4>x\geq 0;\ x<0$
    
    $|2-|x|| = 2+|x|$
    \begin{itemize}
      \item $x\geq 2$: $|x|=x$, $|2-|x|| = -2+x \Rightarrow -2+x=2+x \Rightarrow 2=-2$ absurd.
      \item $2>x\geq 0$: $|x|=x$, $|2-|x||=2-x \Rightarrow x-x=x+x \Rightarrow x=-x \Rightarrow x=0$
      \item $0>x\geq -2$: $|x|=-x$, $|2-|x|| = 2+x \Rightarrow 2-x=2+x \Rightarrow x=0$
      \item $x<-2$: $|x|=-x$, $|2-|x|| = -2+x \Rightarrow -2+x=2-x \Rightarrow 2x=4 \Rightarrow x=2$ absurd.
    \end{itemize}
    The solution is $x=0$.
  \end{enumerate}
\end{exmp}

\subsection*{Some applications of the axiomatic definition of $\mathbb{R}$}

\textbf{1.7}
\begin{enumerate}[label=\emph{\alph*})]
  \item $ax=a$ and $a\neq 0 \Rightarrow x = \frac{a}{a} = 1$
  \item ${(x+y)}^2 = (x+y)(x+y) = x^2 + 2xy + y^2$
  \item $(x+y)(x-y) = x^2 + xy - xy - y^2 = x^2 - y^2$
  \item $x^2 = y^2 \Rightarrow x = \pm \sqrt{x^2} \Rightarrow \begin{cases}
    x = +|y| \\
    x = -|y|
  \end{cases} \Rightarrow x = \pm y$
  \item $(x-y)(x^2 + xy + y^2) = x^3 + x^2y + x y^2 - x^2y - x y^2 - y^3 = x^3 - y^3$
  \item $\displaystyle (x-y) (\sum_{i = 0}^{n-1} x^{n-(i+1)}y^i) = \sum_{i = 0}^{n-1} x^{n-i}y^i - \sum_{i = 0}^{n-1} x^{n-(i+1)}y^{i+1} = x^n + \sum_{i = 1}^{n-1} x^{n-i} y^i - \sum_{i = 0}^{n-2} x^{n-(i+1)}y^{i+1} - y^n = x^n + \sum_{i = 1}^{n-1} x^{n-i} y^i - \sum_{i = 1}^{n-1} x^{n-i}y^{i} - y^n  = x^n - y^n$
\end{enumerate}

\textbf{1.8} $0<a<b$
\begin{enumerate}[label=\emph{\alph*})]
  \item $\displaystyle 0 < \frac{a+b}{2} < \sqrt{\frac{a^2+b^2}{2}} \Rightarrow \frac{a^2+b^2+2ab}{4} < \frac{a^2+b^2}{2} \Rightarrow 2ab < a^2+b^2 \Rightarrow 0 < a^2 + b^2 - 2ab \Rightarrow 0 < {(a-b)}^2$
  \item $\sqrt{ab} < \frac{a+b}{2} \Rightarrow 2\sqrt{ab} < a + b \Rightarrow 4ab < a^2 + b^2 + 2ab \Rightarrow 2ab < a^2+b^2$
  \item $\frac{2ab}{a+b} < \sqrt{ab} \Rightarrow 2ab < \sqrt{ab}(a+b) \Rightarrow 4a^2b^2 < ab {(a+b)}^2 \Rightarrow 4ab < a^2 + b^2 + 2ab$
\end{enumerate}

\textbf{1.8}
\begin{enumerate}[label=\emph{\alph*})]
  \item $a \leq b$ and $\forall \epsilon >0,\ a\leq b \leq a+\epsilon$

  \begin{proof}
    If $a<b \Rightarrow 0<b-a \Rightarrow \exists \epsilon_b >0\ |\ \epsilon_b < b-a \Rightarrow a + \epsilon_b < b$ with $\epsilon_b > 0 \Rightarrow$ absurd $\Rightarrow a=b$.
  \end{proof}
  
  \item $a\leq b$ and $\forall \epsilon > 0\quad b-\epsilon \leq a \leq b$.

  \begin{proof}
    If $a<b \Rightarrow a-b < 0 \Rightarrow \exists \epsilon_a > 0\ |\ a-b < -\epsilon_a \Rightarrow a<b-\epsilon_a,\ \epsilon_a > 0$, absurd $\Rightarrow a=b$
  \end{proof}
\end{enumerate}

\subsection*{Boundaries of real subsets}
\boldmath
\begin{defn}
  Let $A \subseteq \mathbb{R}$. $A$ is called \textbf{bounded from above} if $\exists k\in\mathbb{R}\ |\ \forall a\in A,\ a\leq k$. In this case, $k$ is called upper bound of $A$.

  $A$ is called \textbf{bounded from below} if $\exists k\in\mathbb{R}\ |\ \forall a\in A,\ a\geq k$. $k$ is called lower bound.

  If $A$ has both upper and lower bounds, then $A$ is called a bounded set.
\end{defn}
\unboldmath

\begin{exmp}
  $(0,+\infty)$ is bounded from below by any $k \leq 0$ but not bounded from above.
\end{exmp}

\begin{exmp}
  If we have the following set: $A=[0, \ 3)$, we have this interval of upper bounds: $[3, \ \infty]$ and this one of lower bounds: $(-\infty,\ 0]$

  Then infimum: 0 $\Rightarrow$ as $0 \in A \Rightarrow 0$ is minimum.

  Supremum: 3 $\Rightarrow$ there does not exist maximum!
\end{exmp}

\begin{defn}
  $A \subseteq \mathbb{R},\ \alpha$ is called \textbf{supremum} of $A$ (or least upper bound) if:
  \begin{itemize}
    \item $\alpha$ is an upper bound of $A$.
    \item $\forall k\in\mathbb{R}$ upper bound of $A$, $\alpha \leq k$.
  \end{itemize}
  We denote $\alpha = \sup A$.
\end{defn}

\begin{defn}
  $A \subseteq \mathbb{R},\ \beta$ is called \textbf{infimum} of $A$ (or greatest lower bound) if:
  \begin{itemize}
    \item $\beta$ is an upper bound of $A$.
    \item $\forall k\in\mathbb{R}$ lower bound of $A$, $\beta \geq k$.
  \end{itemize}
  We denote $\beta = \inf A$.
\end{defn}

\begin{exmp}
  $A = [3,5],\quad \sup A = 5$ and $\inf A = 3$.
\end{exmp}

\begin{proposition}
  \begin{itemize}
    \item If $A \subseteq \mathbb{R}$ is bounded from above, then $A$ admits a supremum.
    \item If $A \subseteq \mathbb{R}$ is bounded from below, then $A$ admits a infimum.
  \end{itemize}
\end{proposition}

\begin{defn}
  Let $A \subseteq \mathbb{R}$
  \begin{itemize}
    \item If $\sup A$ exists and $\sup A \in A$, then $\sup A$ is called \textbf{maximum} of $A$ and is denoted $\max A$.
    \item If $\inf A$ exists and $\inf A \in A$, then $\inf A$ is called \textbf{minimum} of $A$ and is denoted $\min A$.
  \end{itemize}
\end{defn}

\begin{exmp}
  $A = [3,4),\quad \min A = \inf A = 3. \quad \sup A = 4 \notin A \Rightarrow \max A$ does not exist.
\end{exmp}

\textbf{1.10}: $A$ bounded and $A_0 \subseteq A$.
\begin{itemize}
  \item \textbf{From above}: If $A_0$ is not bounded from above (b.f.a.) $\forall k\in\mathbb{R},\ \exists a_0 \in A_0\ |\ a_0 >k \Rightarrow \forall k\in\mathbb{R},\ \exists a_0 \in A\ |\ a_0>k \Rightarrow A$ is not b.f.a. $\Rightarrow$ absurd.

  Furthermore, $\forall a\in A_0,\ a\in A$ and $a\leq \sup A \Rightarrow \sup A$ is an upper bound of $A_0 \Rightarrow \sup A \geq \sup A_0$.
  \item \textbf{From below}: same idea.
\end{itemize}

\textbf{1.10}: $A,B\subset \mathbb{R}$ bounded.
\[
  A+B = \{x\in\mathbb{R}\ |\ x=a+b,\ a\in A,b\in B\}
\]

\begin{itemize}
  \item $\forall x\in A+B,\ x=a+b \leq \sup A + \sup B$ since $a\leq \sup A \wedge b\leq\sup B \Rightarrow \sup A+ \sup B$ is an upper bound of $A+B$.

  Let $k \in\mathbb{R}\ |\ \forall x\in A+B, \quad x\leq k \Rightarrow \forall a\in A, b\in B \quad a+b\leq k \Rightarrow \forall a\in A\quad a\leq k-b \Rightarrow \sup A \leq k-b,\ \forall b\in B \Rightarrow \forall b\in B, b \leq k-\sup A \Rightarrow \sup B \leq k - \sup A \Rightarrow \sup A + \sup B \leq k$.
  Thus $\sup A+B = \sup A + \sup B$.
  \item Same idea.
  \item $\alpha > 0:\ \alpha A = \{x\in\mathbb{R}\ |\ x=\alpha a,\ a\in A\}$
  \begin{itemize}
    \item $\forall x\in A,\ x=\alpha a \leq \alpha \sup A$, since $a\leq \sup B$
    \item Let $k\in\mathbb{R}\ |\ \forall x\in \alpha A, \quad x\leq k \Rightarrow \forall a\in A,\ \alpha a\leq k$.
  \end{itemize}
  Thus $\alpha \sup A = \sup \alpha A$.

  $\inf \alpha A =$ same idea.
  \item Same idea but $\alpha < 0$ (change in inequalities)
  \begin{itemize}
    \item $\forall x\in A,\ x=\alpha A \geq \alpha \sup A$.
    \item $\forall a\in A,\ \alpha a\geq k \Rightarrow \forall a\in A,\ a\leq \frac{k}{\alpha} \Rightarrow \sup A\leq \frac{k}{\alpha} \Rightarrow \alpha \sup A \geq k$.
  \end{itemize}
\end{itemize}

\textbf{1.12}:
\begin{enumerate}[label=\emph{\alph*})]
  \item $A = \{2;2.2;2.22;2.222;\ldots\}$
  \begin{itemize}
    \item $\forall a\in A, \forall x\in(-\infty,2], x\leq a \Rightarrow (-\infty, 2]$ is the set of lower bounds.
    \item $\inf A=2$ (the greatest lower bound).
    \item $\inf A \in A \Rightarrow \min A = 2$.
    \item $\forall a\in A,\ \forall x\in [2.\wideparen{2}, \infty),\ x\geq a \Rightarrow [2.\wideparen{2}, \infty)$ set of upper bounds.
    \item $\sup A = 2.\wideparen{2}$ least upper bound.
    \item $\sup A \notin A$, max does not exist.
  \end{itemize}
  \item $\forall r\in\mathbb{R}, z=\text{floor}(r)+1 \in\mathbb{Z}$ and $z>r$.

  $z=\text{floor}(r)-1 \in\mathbb{Z}$ and $z<r \Rightarrow \mathbb{Z}$ does not admit lower bounds.
  \item Study the roots.
  \item $2r^3-1 < 15 \Rightarrow r^3 < 8 \Rightarrow r < 2$
  \begin{itemize}
    \item $[2,\infty)$ is the set of upper bounds.
    \item $\sup A = 2$ and $\sup A \notin A$.
    \item No lower bounds.
  \end{itemize}
  \item $x^2-x-2 < 0$
  \begin{itemize}
    \item $(-\infty,-1]$ set of lower bounds and $[2,\infty)$ set of upper bounds.
    \item $\sup A = 2\in\mathbb{R},\quad \inf A = -1\in\mathbb{Q}$
    \item No max nor min.
    \item Roots are $-1$ and $2$.
  \end{itemize}
\end{enumerate}




% New Unit ==========================================================================================================================================
% New Unit ==========================================================================================================================================
% New Unit ==========================================================================================================================================
% New Unit ==========================================================================================================================================
% New Unit ==========================================================================================================================================
% New Unit ==========================================================================================================================================
% New Unit ==========================================================================================================================================
% New Unit ==========================================================================================================================================
% New Unit ==========================================================================================================================================









\chapter{Sequences and Series}

\section{Sequences}
\textbf{*Basical Concepts:}

\begin{defn} A sequence of real numbers is a map (or function) \newline
  $f 
  \begin{alignedat}{2}
     &  :\mathbb{N} \rightarrow \mathbb{R}\\
     &  :n \rightarrow f(n)=x_n \\ 
  \end{alignedat}$ \newline
  It can be denoted by $(x_n)^b_{n=a}$ where $a$ is the lower value of $n$ or $-\infty$ and $b$ is the upper value or $+\infty$
\end{defn}

\begin{exmp}
  $$(\frac{1}{2^n})^3_{n=1} = \{\frac{1}{2}, \frac{1}{4}, \frac{1}{8}\}$$
  $$(2^n)^\infty_{n=0} = \{1,2,4,8, ...\}$$
\end{exmp}

\begin{defn} We call recurvise sequence to sequences whose elements are related to 
  previous elements in a straight-forward way. \textbf{i.e.} $x_{n+1}=f\{x_n,...x_1\}$
\end{defn}

\begin{exmp}
  $$x_0=2 \ \Rightarrow \ x_n=(x_{n-1})^2+1 \Rightarrow \ x_0=2,\  x_1=5,\  x_2=26 \ ...$$ \newline
\end{exmp}

\textbf{*Limit}

\begin{defn} We call $x \in \mathbb{R}$ the limit of $(x_n)^{+\infty}_{n=a}$ if: \newline
  \text{\hspace{1cm}} $\forall \varepsilon >0, \exists n_0 \in \mathbb{N}$ such that $\forall n >n_0$ we have $\abs*{x_n-x}< ???$ \newline
  In this case $(x_n)$ is said to converge to the limite $x$ and we denote:
  $$\lim_{n\rightarrow +\infty}x_n=x$$
  
\end{defn}

\begin{defn} $(x_n)^{+\infty}_{n=a}$ is said to converge to \dots
  \begin{enumerate}[label=\alph*]
    \item Infinity, if: $\forall n >0, \exists n_0 \in \mathbb{N}$ such that $\forall n >n_0 \Rightarrow x_n>n$
    \item Minus Infinity, if: $\forall n<0, \exists n_0 \in \mathbb{N}$ such that $\forall n >n_0 \Rightarrow x_n < n$ \newline
  \end{enumerate}
\end{defn}

\begin{defn} If a sequence $(x_n)^{+\infty}_{n=a}$ doesn't converge, we denote that it \textbf{diverges}
\end{defn}

\begin{exmp} \
  \begin{enumerate}[label=\alph*]
    \item $\lim_{n \rightarrow +\infty} \frac{1}{n^2+1}=0$ \newline
    $\text{Let } \varepsilon >0, \text{ (we want) }\abs*{\frac{1}{n^2+1}-0}< \varepsilon$
    $$\Rightarrow \frac{1}{n^2+1}< \varepsilon \Rightarrow \frac{1}{\varepsilon}-1 < n^2 \Rightarrow \sqrt{\frac{1}{\varepsilon}-1}<n$$
    $$\Rightarrow N=ceil(\sqrt{\frac{1}{\varepsilon}-1}), \forall n >N \ \frac{1}{n^2+1}< \varepsilon$$
    $$\Rightarrow 0 \text{ is the limit of } \frac{1}{n^2+1}$$
    $\varepsilon = 1e^{-3} \Rightarrow N=32 \Rightarrow \frac{1}{32^2+1}=0.9710^{-3}$

    \item  $\abs*{\frac{3n-1}{4n}-\frac{3}{4}} < \varepsilon$ \newline
    $\text{As } \frac{3n-1}{4n} < \frac{3n}{4n} \Rightarrow \frac{-3n-1}{4n} + \frac{3}{4} < \varepsilon$
    $$\Rightarrow -(3n-1) < (\varepsilon -\frac{3}{4})4n \Rightarrow 1< ((\varepsilon -\frac{3}{4})4+3)n$$
    $$\Rightarrow -\frac{1}{4(\varepsilon -\frac{3}{4})+3} < n$$
    $\text{With } \varepsilon=1e^{-3} \Rightarrow N=250$
  \end{enumerate}
\end{exmp}

\begin{defn} A sequence $(x_n)_{n\in \mathbb{N}}$ is said to be an alternating sequence if it is of the form: $x_n=(-1)^n a_n$ with $a_n>0$ or $<0 \ \forall n \in \mathbb{N}$
\end{defn}

\begin{exmp} If $x_n=(-1)^n a_n$ with $a_n>0$ or $a_n<0$ is convergent 
  $$\Rightarrow \lim_{n \rightarrow +\infty} x_n=0$$
\end{exmp}

\begin{proof}
  $lim_{n \rightarrow +\infty} x_n =L$ with $L \neq 0$ absurd: for the sake of simplicity $L>0$ and $a_n>0$ (other cases lead to a similar proof) \newline
  $\text{Let } \varepsilon=L-0, \forall N_0 \in \mathbb{N}, \exists n_1=$
  $\begin{alignedat}{2}
    &N_0+2 \text{ if $N_0$ is odd}\\
    &N_0+1 \text{ if $N_0$ is even}\\
  \end{alignedat}$ \newline
  $\text{Such that } x_{n1}=(-1)^{n1}a_n<0$
  $$\Rightarrow \abs*{x_{n-1}-L}= \abs*{a_n+L}> \varepsilon \text{ absurd!}$$
  $$\Rightarrow L\neq 0 \text{ cannot be limit } \Rightarrow \text{ if } x_n \text{ converges, $0$ must be the limit}$$
\end{proof}

\begin{prop}
  $\text{If } (x_n)^{+\infty}_{n=a}$ is convergent, the limit is unique
\end{prop}

\begin{exmp} $\lim_{n \rightarrow +\infty} \ x_n=x$
  \begin{enumerate}[label=\alph*]
    \item $a>x \Rightarrow a-x=$
  \end{enumerate}
  
\end{exmp}

\chapter{Continious Functions}

\section{Basical concepts of function}

\begin{exmp}
  Domain of: \\
  a) \ \ $f(x) = \sqrt{1-x^2} \ \ Dom\sqrt{*} = [0, +\infty]$

  we want $ 1-x^2 \geq 0 \rightarrow x \in (-\infty, 1] \cup [1, +\infty) \rightarrow Dom f = (-\infty, -1) \cup [1, +\infty] $
\end{exmp}



\end{document}